%%%%%%%%%%%%%%%%%%%%%%%%%%%%%%%%%%%%%%%%%%%%%%%%%%%%%%%%%%%%%%%%%%%%%%%%%%%%%%%%%%%%%%%%%%%%%%%%%%%%%%%%%%%%%%%%%%%%%%%%%%%%%%%%%%%%%%%%%%%%%%%%%%%%%%%%%%%%%%%%%%%%%%%%%%%%%%%%%%%%%%%%%%%%%%%%%%%%%%
\documentclass{article}
\usepackage[top = 0.25in, bottom = 0.25in, left = 0.25in, right = 0.25in]{geometry}	% 设置页面布局
\usepackage{ulem} 		% 双下划线 删除线 等
\usepackage{framed}     % 实现方框效果
\usepackage{amsmath}	% 公式排版
\usepackage{amssymb}	% 更多的公式
\usepackage{wrapfig}	% 图片环绕
\usepackage{fontspec} 	% 设置字体
\usepackage{graphicx}	% 图片
\usepackage{hyperref}	% 超链接
\usepackage{listings}   % 代码高亮
\usepackage{setspace}	% 设置间距
\usepackage{blindtext}	% 测试文字
\usepackage[strict]{changepage} % 提供一个 adjustwidth 环境
\usepackage[dvipsnames,svgnames]{xcolor} % 需要在tcolorbox 前面定义
\usepackage{pgfplots}   % 化2D3D的矢量图
\usepackage{tcolorbox}
\setmainfont{Nirmala UI}
\newfontfamily\consolas{Consolas}
%%%%%%%%%%%%%%%%%%%%%%%%%%%%%%%%%%%%%%%%%%%%%%%%%%%%%%%%%%%%%%%%%%%%%%%%%%%%%%%%%%%%%%%%%%%%%%%%%%%%%%%%%%%%%%%%%%%%%%%%%%%%%%%%%%%%%%%%%%%%%%%%%%%%%%%%%%%%%%%%%%%%%%%%%%%%%%%%%%%%%%%%%%%%%%%%%%%%%%
\setlength{\parindent}{0pt} 	% 首行无空格
\graphicspath{{./figures/}}      % 设置图片的路径

\lstset{
    % background
    backgroundcolor=\color[RGB]{245,245,244},
    % fonts' style and color
    basicstyle = {
        \fontspec{Consolas}
    },
    keywordstyle = {%
        \color[RGB]{0,0,0}%
        \fontspec{Consolas}%
    },%
    commentstyle = {%
        \color{red!30!green!90!blue!100}%
        \fontspec{Consolas Italic}%
    },%
    stringstyle = {%
        \color[RGB]{0,0,0}%
        \fontspec{Consolas}%
    },%
    numberstyle = {%
        \color[RGB]{0,0,0}%
        \fontspec{Consolas}%
    },%
    % tab
    tabsize=2,
    % nums of code
    % numbers=left,
    % display space or tab
    showstringspaces=false,showtabs=true,
}
%%%%%%%%%%%%%%%%%%%%%%%%%%%%%%%%%%%%%%%%%%%%%%%%%%%%%%%%%%%%%%%%%%%%%%%%%%%%%%%%%%%%%%%%%%%%%%%%%%%%%%%%%%%%%%%%%%%%%%%%%%%%%%%%%%%%%%%%%%%%%%%%%%%%%%%%%%%%%%%%%%%%%%%%%%%%%%%%%%%%%%%%%%%%%%%%%%%%%%
% 注意行末需要把空格注释掉,不然画出来的方框会有空白竖线
\tcbuselibrary{most}
\definecolor{formalshade}{rgb}{0.95,0.95,1}         % 灰蓝色
\definecolor{celadon}{rgb}{0.80, 0.88, 0.69}        % 浅绿色
\definecolor{azure}{rgb}{0.0, 0.5, 1.0}             % 浅蓝色
\newenvironment{formal}{%
\def\FrameCommand{%
\hspace{1pt}%
{\color{Gray}\vrule width 2pt}%
{\color{formalshade}\vrule width 4pt}%
\colorbox{formalshade}%
}%
\MakeFramed{\advance\hsize-\width\FrameRestore}%

\noindent\hspace{-4.55pt}%
\begin{adjustwidth}{}{7pt}%
\vspace{2pt}\vspace{2pt}%
}%
{%
\vspace{2pt}\end{adjustwidth}\endMakeFramed%
}
% 题干
\newenvironment{problem}{%
\def\FrameCommand{%
\hspace{1pt}%
\colorbox{celadon}%
}%
\MakeFramed{\advance\hsize-\width\FrameRestore}%
\noindent\hspace{-4.55pt}%
\begin{adjustwidth}{}{7pt}%
\vspace{2pt}\vspace{2pt}%
}%
{%
\vspace{2pt}\end{adjustwidth}\endMakeFramed%
}
% 代码
\newenvironment{code}{%
\def\FrameCommand{%
\hspace{1pt}%
\colorbox{formalshade}%
}%
\MakeFramed{\advance\hsize-\width\FrameRestore}%
\noindent\hspace{-4.55pt}%
\begin{adjustwidth}{}{7pt}%
\vspace{2pt}\vspace{2pt}%
}%
{%
\vspace{2pt}\end{adjustwidth}\endMakeFramed%
}
%%%%%%%%%%%%%%%%%%%%%%%%%%%%%%%%%%%%%%%%%%%%%%%%%%%%%%%%%%%%%%%%%%%%%%%%%%%%%%%%%%%%%%%%%%%%%%%%%%%%%%%%%%%%%%%%%%%%%%%%%%%%%%%%%%%%%%%%%%%%%%%%%%%%%%%%%%%%%%%%%%%%%%%%%%%%%%%%%%%%%%%%%%%%%%%%%%%%%%

\title{\huge Algorithm Improvement Course}
\author{\huge Onnes}
\date{\huge 2022/12/10}

%%%%%%%%%%%%%%%%%%%%%%%%%%%%%%%%%%%%%%%%%%%%%%%%%%%%%%%%%%%%%%%%%%%%%%%%%%%%%%%%%%%%%%%%%%%%%%%%%%%%%%%%%%%%%%%%%%%%%%%%%%%%%%%%%%%%%%%%%%%%%%%%%%%%%%%%%%%%%%%%%%%%%%%%%%%%%%%%%%%%%%%%%%%%%%%%%%%%%%
\begin{document}
\maketitle
\tableofcontents
%%%%%%%%%%%%%%%%%%%%%%%%%%%%%%%%%%%%%%%%%%%%%%%%%%%%%%%%%%%%%%%%%%%%%%%%%%%%%%%%%%%%%%%%%%%%%%%%%%%%%%%%%%%%%%%%%%%%%%%%%%%%%%%%%%%%%%%%%%%%%%%%%%%%%%%%%%%%%%%%%%%%%%%%%%%%%%%%%%%%%%%%%%%%%%%%%%%%%%
\newtcolorbox{blue}[2][]{colbacktitle=Gray, colback=formalshade,coltitle=black,title={#2},fonttitle=\bfseries,#1}   % 定理
\newtcolorbox{red}[2][]{colbacktitle=Red, colback=formalshade,coltitle=black,title={#2},fonttitle=\bfseries,#1}    % 注意
%%%%%%%%%%%%%%%%%%%%%%%%%%%%%%%%%%%%%%%%%%%%%%%%%%%%%%%%%%%%%%%%%%%%%%%%%%%%%%%%%%%%%%%%%%%%%%%%%%%%%%%%%%%%%%%%%%%%%%%%%%%%%%%%%%%%%%%%%%%%%%%%%%%%%%%%%%%%%%%%%%%%%%%%%%%%%%%%%%%%%%%%%%%%%%%%%%%%%%
\clearpage
\section{1.dynamic programming} {

\subsection{1.digital triangle model} {

\subsubsection{Acwing 1015.picking peanuts} {

\paragraph{question} {
\href{https://www.acwing.com/problem/content/1017/}{Acwing 1015.picking peanuts}\\
time limit per test : 1 second\\
memory limit per test: 64 megabytes\\
input:standard input\\
output:standard output\\

\begin{problem}
\vskip -0.24 cm
\hskip -0.10 cm
{\color{Green}\noindent\rule[0.25\baselineskip]{20.29cm}{1pt}}\\
Hello Kitty wants to pick some peanuts for her favorite Mickey Mouse.\\
She came to a rectangular peanut field with grid-like roads (pictured below), entering from the northwest corner and exiting from the southeast corner.\\
At the intersection of each road in the field, there is a peanut seedling with several peanuts on it, and all the peanuts on it can be picked off after passing through a peanut seedling.\\
Hello Kitty can only go east or south, not west or north.\\
Ask Hello Kitty how many peanuts can be picked at most.\\
\vskip -0.24 cm
\hskip -0.10 cm
{\color{Green}\noindent\rule[0.25\baselineskip]{20.29cm}{1pt}}
\vskip -0.28 cm
\end{problem}
% % normal
% \begin{figure}[h]\centering
% \includegraphics[width=8cm]{fangge.jpg}
% \end{figure}
% % surround
% \begin{wrapfigure}{r}{0.25\textwidth} %this figure will be at the right
% % caption on picture
% \includegraphics[width=0.25\textwidth]{mesh.png}
% \end{wrapfigure}






input
\begin{formal}
The first line is an integer T, representing how many sets of data there are.\\
Next is the T group of data.\\
The first line of each set of data is two integers, representing the number of rows R and the number of columns C of peanut seedlings.\\
The next R row of data in each group of data describes the situation of each row of peanut seedlings in sequence from north to south. There are C integers in each row of data, and the number M of peanuts on each peanut seedling in the row is described in order from west to east.\\
\end{formal}
output
\begin{formal}
For each set of input data, output a line containing the maximum number of peanuts that Hello Kitty can pick.
\end{formal}
range
\begin{formal}
\vskip -0.5 cm
\begin{flalign}
\notag
&\ \mathtt{1\leq T\leq 100}\\
\notag
&\ \mathtt{1\leq R,~C\leq 100}\\
\notag
&\ \mathtt{0\leq M\leq 100}&
\end{flalign}
\vskip -0.5 cm
\end{formal}
\begin{tcolorbox}[title={input}]
2\\
2 2\\
1 1\\
3 4\\
2 3\\
2 3 4\\
1 6 5
\end{tcolorbox}
\begin{tcolorbox}[title={output}]
8\\
16
\end{tcolorbox}
}

\clearpage
\paragraph{idea} {
~\\
$~~~~~~~~~~$
\begin{blue}{Theorem 1}
\vskip -0.5 cm
\begin{flalign}
\notag
&\mathtt{x^{2}+y^{2}=r^{2}}&
\end{flalign}
\end{blue}
}

\clearpage
\paragraph{code} {

\begin{lstlisting}[language={C}]
#include <bits/stdc++.h>

using namespace std;

const int N = 1e5 + 10;
int a[N];

int main() {
    int n; // input
    scanf("%d", &n);
    puts("Hello world!");
    for (int i = 0; i < n; i ++ ) {
        printf("%d", a[i]);
    }
    return 0;
}
\end{lstlisting}

}

}

}
%%%%%%%%%%%%%%%%%%%%%%%%%%%%%%%%%%%%%%%%%%%%%%%%%%%%%%%%%%%%%%%%%%%%%%%%%%%%%%%%%%%%%%%%%%%%%%%%%%%%%%%%%%%%%%%%%%%%%%%%%%%%%%%%%%%%%%%%%%%%%%%%%%%%%%%%%%%%%%%%%%%%%%%%%%%%%%%%%%%%%%%%%%%%%%%%%%%%%%
\end{document}
%%%%%%%%%%%%%%%%%%%%%%%%%%%%%%%%%%%%%%%%%%%%%%%%%%%%%%%%%%%%%%%%%%%%%%%%%%%%%%%%%%%%%%%%%%%%%%%%%%%%%%%%%%%%%%%%%%%%%%%%%%%%%%%%%%%%%%%%%%%%%%%%%%%%%%%%%%%%%%%%%%%%%%%%%%%%%%%%%%%%%%%%%%%%%%%%%%%%%%